\chapter{Hardware}
    The hardware side consists of two mechanically connected but kinematically independent goniometer stages, static mounting components and an electrical part.
    The electronics are subdivided into the power supply, motion and logic handling and an interface between XRS/detector/motion system and the logic.

    \section{Mechanics}
        Mechanically the main challenge is to build two co-axial and co-planar but kinematically independent operating goniometer stages.
        One to move the detector and one to move the specimen about there mutual center of rotation as shown in~\cref{fig:rough sketch of the motion system}.
        \begin{figure}[h]
            \centering
            \includesvg[width=.6\textwidth]{drawings/sketch_stages}%
            \label{fig:rough sketch of the motion system}
            \caption[Rough sketch of the motion system.]{Rough sketch of the motion system.}
        \end{figure}
        The Bulb depicted in~\cref{fig:rough sketch of the motion system} represents the XRS, the eye represents the detector and the specimen to analyze is positioned in the center\footnote{Clearly, an X-Ray source most likely will not be found in the fashion of an actual incandescent light bulb. Here, it merely serves as a visual shortcut representing a photon source. Furthermore, an actual eye performs rather poor at detecting \(\gamma\)-radiation. Thus, it, as well, shall serve the purpose of representation.}.
        The center stage varies the angle~\(\theta\) which is the angle spun between the incident beam path and the plane of the specimen.
        The detector stage~-~the outer circle in~\cref{fig:rough sketch of the motion system}~-~gets positioned along an angle~\(\varphi\).\par\medskip

        While both stages need to be capable of accurate and repeatable positioning, the detector stage needs to move and carry the weight of the actual detector and its supporting structure.
        

        \subsection{Sample Stage}
            \begin{figure}[t]
                \centering
                \includesvg[width=.8\textwidth]{drawings/X-MAGIX_full_sample}
                \label{fig:xmagix full sample highlightes}
                \caption[Full motion system. Sample stage blue highlighted.]{Full motion system. Sample stage blue highlighted.}
            \end{figure}
            \lipsum
            \begin{figure}[t]
                \centering
                \includesvg[height=.5\textheight]{drawings/sample_stage_-_exploded}
                \label{fig:sample stage exploded}%
                \caption[Sample stage exploded view.]{Sample stage exploded view.}
            \end{figure}
            \lipsum
            \begin{figure}[t]
                \centering
                \includesvg[height=.5\textheight]{drawings/sample_stage_-_exploded_section}
                \label{fig:sample stage exploded section}%
                \caption[Sample stage exploded view -- sectioned.]{Sample stage exploded view -- sectioned.}
            \end{figure}
        \subsection{Detector Stage}
            \begin{figure}[h]
                \centering
                \includesvg[width=.8\textwidth]{drawings/X-MAGIX_full_detector}
                \label{fig:xmagix full detector highlightes}
                \caption[Full motion system. Detector stage blue highlighted.]{Full motion system. Detector stage blue highlighted.}
            \end{figure}
            \lipsum
            \begin{figure}[t]
                \centering
                \includesvg[height=.5\textheight]{drawings/detector_stage_-_exploded}
                \label{fig:detector stage exploded section}%
                \caption[Detector stage exploded view.]{Detector stage exploded view.}
            \end{figure}
            \lipsum
        \subsection{Base and X-Ray Source}
            \lipsum

    \section{Electronics}
        \subsection{Kinematics}

        \subsection{Raspberry Pi to X-Ray Source Interface}