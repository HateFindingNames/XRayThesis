\chapter{Hardware}
    The hardware side consists of two mechanically connected but kinematically independent goniometer stages, static mounting components and an electrical part.
    The electronics are subdivided into the power supply, motion and logic handling and an interface between XRS/detector/motion system and the logic.

    \section{Mechanics}
        Mechanically the main challenge is to build two co{-}axial and co{-}planar but kinematically independent operating goniometer stages.
        One to move the detector and one to move the specimen about there mutual center of rotation as shown in~\cref{fig:rough sketch of the motion system}.
        \begin{figure}[h]
            \centering
            \includesvg[width=.6\textwidth]{drawings/sketch_stages}%
            \caption[Rough sketch of the motion system.]{Rough sketch of the motion system.}%
            \label{fig:rough sketch of the motion system}%
        \end{figure}
        The light bulb depicted in~\cref{fig:rough sketch of the motion system} represents the XRS, the eye represents the detector and the specimen to analyze is positioned in the center\footnote{Clearly, an X{-}Ray source most likely will not be found in the fashion of an actual incandescent light bulb. Here, it merely serves as a visual shortcut representing a photon source. Furthermore, an actual eye performs rather poor at detecting \(\gamma\){-}radiation. Thus, it, as well, shall serve the purpose of representation.}.
        The center stage varies the angle~\(\theta\) which is the angle spun between the incident beam path and the plane of the specimen.
        The detector stage~--~the outer circle in~\cref{fig:rough sketch of the motion system}~--~gets positioned along an angle~\(\varphi\).\par\medskip

        While both stages need to be capable of accurate and repeatable positioning, the detector stage needs to accelerate and carry the mass of the actual detector and its supporting structure.

        The rise of desktop 3D{-}printing and the growing communities around highly sophisticated open source/hardware machines made components for high precision yet low cost motion systems in the low to lower{-}mid power range more available than ever.
        Hence, common 3D{-}printer parts where used where applicable.
        To achieve accuracy a gearing system utilizing stepper motors, purpose made timing pulleys and timing belts with a GT3 profile and~\qty{2}{\milli\metre} pitch are chosen\cite{Manual.POWERGRIPGT3,Manual.LIGHTPOWERPRECISION}.
        
        Besides other factors, to maintain repeatability the easiest ones to achieve are reduced springiness in the whole motion system by choosing good quality, glass{-}fiber lined timing belts and eliminating slip during changes of torque e.g. direction changes and/or acceleration and deceleration by means of preloading the drive system.\par\medskip

        A detailed description of the in this work proposed solutions are discussed in~\cref{sec:sample stage,,sec:detector stage}.
        
        \subsection{Sample Stage}\label{sec:sample stage}
            \Cref{fig:xmagix full sample highlightes} shows the main kinematic assembly and the sample stage with drive highlighted in light blue.
            \begin{figure}[ht]
                \centering
                \includesvg[width=.6\textwidth]{drawings/X-MAGIX_full_sample}%
                \caption[Full motion system. Sample stage blue highlighted.]{Full motion system. Sample stage highlighted in light blue.}%
                \label{fig:xmagix full sample highlightes}%
            \end{figure}%

            The center part of the main assembly is the slewing ring \textit{PRT{-}01{-}60} by \textsc{igus}\cite{Manual.IglidePRTPolymerSlewingRings.} depicted as no. 6 in the section view in~\cref{fig:sample stage exploded section}.
            Its relatively large center hole gives room to place a smaller bearing assembly co{-}axially to its own axis of rotation.\par\medskip

            The stationary part of slewing ring provides two circular patterns of tapped bores fitting 10 M5 fasteners each.
            The holes along the inner circle are free to use.
            \textit{Stator center} (no. 7) is inserted into the slewing ring and hold in place by seven M5x10 bolts (no. 13).
            Hereafter and in that sequence, the \textit{stator center spacer} (no. 5) and ball bearing \textit{W 61708{-}2RS1} get inserted into the free space from the top.
            One secures the ball bearing press{-}fitting six of the \textit{stator locks} into there respective slots\footnote{ "Abusing" an ordinary drill{-}press to~--~well~--~press them in place (turned off!) works perfect.}.
            \begin{figure}[h]
                \centering
                \includesvg[width=.5\textwidth]{pictures/footage/stage_bearings}
                \caption[Detailed photo of the stator center]{Detailed photo of the stator center. Zoomed in a detailed view of the press{-}in{-}place stator center locks (1). At the top and bottom are the \textit{W 61708} bearings visible (2) with the \textit{stator center spacer} in between. The outer part is the \textit{stator center} itself. All parts except the bearings are 3D-printed in ABS.}
                \label{fig:detailed view of the stator center}
            \end{figure}
            With slight force the remaining bearing gets fit in place from the bottom against the \textit{stator center spacer} as well.
            
            Finally, the \textit{sample driven gear} (no. 14) and the \textit{endstop bump sample} (no. 11) together are tightened against the \textit{sample platform} (no. 2) which is placed into the bearings beforehand.
            In order to guide the timing belt later, two bearings stacks are screwed from the bottom into two adjacent tapped holes.

            \begin{figure}[h]
                \centering
                \includesvg[width=.8\textwidth]{drawings/sample_stage_-_exploded_section}%
                \caption[Sample stage exploded view~-~sectioned.]{Sample stage exploded view~-~sectioned. 1: Sample Platform, 2: Sample Driven Gear 80T, 3: W 61708, 4: Stator Center, 5: Stator Center Spacer, 9: M5 Hex nut, 10: M5x25, 13: Washer 5.3, 16: 635{-}2RS, 17: Endstop Bump Sample, 19: M5x55, 20: M5x50, 21: Stator Center Spacer Lock, 22: M5x10, 24: PRT{-}01{-}60.}%
                \label{fig:sample stage exploded section}%
            \end{figure}%
            Not shown in~\cref{fig:sample stage exploded section} is the driving sub{-}assembly for the sample stage.
            Since its assembly is considered straight forward and a more complex version is shown in~\cref{fig:detector drive exploded} an assembled view is found in~\cref{fig:sample drive} for reference.

        \subsection{Detector Stage}\label{sec:detector stage}
            The components comprising the detector stage are depicted in~\cref{fig:xmagix full detector highlightes}.
            Its principle of operation follows the same basic concept~--~a timing belt mechanically couples the driving sub{-}assembly on the right{-}hand side to the movable central platform.
            \begin{figure}[ht]
                \centering
                \includesvg[width=.6\textwidth]{drawings/X-MAGIX_full_detector}%
                \caption[Full motion system. Detector stage blue highlighted.]{Full motion system. Detector stage blue highlighted.}%
                \label{fig:xmagix full detector highlightes}%
            \end{figure}

            Again, the detector stage is subdivided into a driving sub{-}assembly as shown in~\cref{fig:detector drive exploded} and driven one shown in~\cref{fig:detector stage exploded} respectively.
            The driven portion of the detector stage only consists of two main parts mounted to the slewing ring.
            The \textit{detector platform} (no. 4) which is placed on the top at an arbitrary orientation and the \textit{detector driven gear} on the bottom of the slewing ring.
            \textit{M5x25} bolts and \textit{hex nuts} are placed into all holes but the two where the \textit{endstop bump detector} (no. 6) is mounted.
            Here, to accommodate the needed extra length, \textit{M5x30} bolts are used instead.
            The \textit{endstop bump detector} itself is oriented such that its outermost mounting hole is aligned with the hole at the clockwise end of the detector arm.
            To lift the detector to a suitable height above the \textit{sample platform }, \textit{detector spacers} are placed underneath the \textit{detector}.\par\medskip
            \begin{figure}[t]
                \centering
                \includesvg[width=.8\textwidth]{drawings/detector_stage_-_exploded}%
                \caption[Detector stage exploded view.]{Detector stage exploded view. 9: M5 Hex Nut, 10: M5x25, 23: XRS{-}A, 25: Detector Platform, 26: Detector Driven Gear 250T, 29: Endstop Bump Detector, 30: Detector Spacer, 31: M3x25, 42: M5x30.}%
                \label{fig:detector stage exploded}%
            \end{figure}
            
            \Cref{fig:detector drive exploded} shows an exploded view of the detector drive sub{-}assembly.
            The core component, a NEMA17 \num{200} steps per revolution stepper motor (no. 7), is placed on the inside of the housing (no. 27) and held in place by four \textit{M3x6} screws.
            Slotted srew holes allow easier installation of the timing belt and adjustment of tension later on.
            Together with a \textit{hex nut} and the \textit{belt tensioner} (no. 8), a \textit{M5x25} bolt allows to hold the motor in place while tensioning the belt prior to finally tightening the screws at the top.
            Bearing stacks identical to those seen in \cref{fig:sample stage exploded} are put in place to confine the belt path along a \qty{180}{\degree} loop around the \textit{80T} timing pulley (no. 28).
            Having as many teeth of the belt engage the pulley at any given time is preferred as it ensures good torque transmission without belt slipping at a lower overall belt tension and thus lower load on the motors.
            \begin{figure}[h]
                \centering
                \includesvg[height=.5\textheight]{drawings/detector_drive_-_exploded}
                \caption[Detector stage drive~-~exploded view.]{Detector stage drive~-~exploded view. 7: 17HM19{-}2004S, 8: Belt Tightener, 9: M5 Hex nut, 10, M5x25, 11: Washer 3.2, 12: M3x6, 13: Washer 5.3, 14: Motor Mount Socket, 16: 635{-}2RS, 27: Motor Mount Detector Drive, 28: GT3 2mm 80T.}
                \label{fig:detector drive exploded}
            \end{figure}

        \subsection{Base and X{-}Ray Source}
            \lipsum

    \section{Electronics}
        On the electronics side, several tasks need to be accomplished.
        Most basic of which is powering each of the systems components.
        The SBC is typically powered via its dedicated USB{-}C port which in turn requires a \qty{5.1}{\volt} supply with a recommended current rating of \qty{3}{\ampere}\cite{Manual.Documentation.RPF}.
        The detectors specifications state, that it needs a \qty{5}{\volt} power supply and draws less than \qty{3}{\watt} which translates to \(<\qty{0.6}{\ampere}\).
        Taken from the datasheet, the XRS is driven by an input voltage of \qty{24}{\volt} and an input current of \qty{1.1}{\ampere}\cite{Manual.MAGPRODataSheet.QD}.
        The two stepper motors are each driven by TMC2209 stepper drivers which in turn accept a relatively broad range of \qtyrange{4.75}{29}{\volt}\cite{Manual.TMC2209Datasheet} as supply voltages.
        The remaining peripherals such as the interface between the SBC and the XRS are directly driven from \qty{3.3}{\volt} and \qty{5}{\volt} provided by the Raspberry Pi's GPIO{-}Header.

        A tabulated summery of the supply power requirements can be found in \cref{tab:power requirements}.\par\medskip
        \begin{table}[h]
            \centering
            \caption[Electrical power requirements of the various installed components]{Electrical power requirements of the various installed components.}%
            \label{tab:power requirements}
            \begin{tabular}{@{}lSS@{}}
                \toprule
                Component&  {Voltage / \unit{\volt}}&    {Current / \unit{\ampere}}\\
                \midrule
                SBC&            5.1&    3.0\\
                Detector&       5.0&    0.6\\
                XRS&            24.0&   1.1\\
                Motor Driver&   24.0&   3.0\\
                \bottomrule
            \end{tabular}
        \end{table}

        Following the above considerations, a \qty{5}{\volt} \qty{25}{\watt} PSU (\textit{RS{-}25{-}5}) and a \qty{24}{\volt} \qty{150}{\watt} PSU (\textit{LRS{-}150{-}24}) are installed.
        Each is able to deliver slightly more power than expected to be necessary.
        Thus, leaving headroom for future extensibility and peaks in power demand.
        The single \qty{24}{\volt}{-}PSU supplying the stepper drivers and the XRS minimizes complexity and reduces costs.\par\medskip

        All wiring~--~where applicable~--~is routed through perforated cable ducting.

        \subsection{Kinematics}

        \subsection{Raspberry Pi to X{-}Ray Source Interface}
            Besides control and monitoring via USB using there USB~-~GUI ``12WattController''

            The following table is taken directly from the technical user guide included with the XRS%
            \footnote{%
            Unfortunately, neither the manufacturer \textsc{Moxtek}, nor the reseller \textsc{Quantum Design Europe} provide freely accessible resources besides the cited data sheets \cites{Manual.MAGPRODataSheet.MOXTEK}{Manual.MAGPRODataSheet.QD}.
            Therefore, this work will only provide contextual necessary transcripts of relevant parts of the included documents.}.
            \begin{table}
                \centering
                \caption[Pinout of the analog control connector of the XRS]{Pinout of the analog control connector of the XRS. Transcribed from the included users manual.}%
                \label{tab:xrs 10pin connector specs}
                \begin{tabular}{@{}llll@{}}
                    \toprule
                    Pin&    Functionn&      I/O Value&                                                          Response/Description\\
                    \midrule
                    1&      HV Enable&      \qtyrange{0.0}{+5.0}{\volt} (Input \(Z = \qty{10}{\kilo\ohm}\))&     \(< \qty{1.0}{\volt} = \text{OFF, } >\qty{4}{\volt} = \text{ON}\)\\
                    2&      Filament Ready& \qtyrange{0.0}{+5.0}{\volt} (Output \(Z = \qty{1}{\kilo\ohm}\))&     \(< \qty{1.0}{\volt} = \text{Not ready, } \qty{5}{\volt} = \text{Ready}\)\\
                    3&      HV Control&     \qtyrange{0.27}{+4.0}{\volt} (Input \(Z = \qty{100}{\kilo\ohm}\))&   \qtyrange{4}{60}{\kilo\volt}\\
                    4&      HV Monitor&      \qtyrange{0.27}{+4.0}{\volt} (Output \(Z = \qty{100}{\ohm}\))&       \qtyrange{4}{60}{\kilo\volt}\\
                    5&      Current Control& \qtyrange{0.04}{+4.0}{\volt} (Input \(Z = \qty{100}{\kilo\ohm}\))&   \qtyrange{10}{1000}{\micro\ampere}\\
                    6&      Current Monitor& \qtyrange{0.0}{+4.0}{\volt} (Output \(Z = \qty{100}{\ohm}\))&        \qtyrange{0}{1000}{\micro\ampere}\\
                    7&      Ground&         Ground&                                                             Ground\\
                    8&      Ground&         Ground&                                                             Ground\\
                    9&      Filament Enable&\qtyrange{0.0}{+5.0}{\volt} (Input \(Z = \qty{10}{\kilo\ohm}\))&     \(< \qty{1.0}{\volt} = \text{OFF, } >\qty{4}{\volt} = \text{ON}\)\\
                    10&     Input Power&    \(\qty{24}{\volt} \pm \qty{5}{\percent}\)&  Input Power\\
                    \bottomrule
                \end{tabular}
            \end{table}

            \begin{figure}[h]
                \centering
                \includesvg[width=.7\textwidth]{pictures/footage/interface_front}
                \caption[Top view of the SBC{-}XRS{-}interface board]{Top view of the SBC{-}XRS{-}interface board. DAC in yellow (1), ADC in purple (2).}%
                \label{fig:top view interface board}
            \end{figure}