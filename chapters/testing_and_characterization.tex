\chapter{Testing and Characterization}\label{chap:testing}
    \section{Data Acquisition}\label{sec:Data Acquisition}
        After some initial calibration,  three spectra were taken from brass, steel and lead.
        Each acquisition run was performed using \texttt{XMagix}'s \texttt{fixedRealtimeRun()} method set to \qty{30}{\second} and \texttt{clearMca} set to \texttt{false}.
        \Cref{tab:default detection parameters} below shows the acquisition values set during each run\footnote{A detailed description of each individual parameter can be found in the~\cite{Software.HandelRelease.2023,Manual.HandelAPIManual.Xiang,Manual.HandelProgrammersGuideMicroDXP.Xiang}.}.

        \begin{table}[!h]
            \centering
            \caption[Detector settings]{Detector settings taken from the technical specifications report coming with the detector. If not stated otherwise, the value stored on the device was used.}%
            \label{tab:default detection parameters}
            \begin{tabular}{@{}lS@{}}
                \toprule
                Parameter&      {Value}\\
                \midrule
                \texttt{parset}&     2.4\\
                \texttt{genset}&     0.0\\
                \texttt{fippi}&     0.0\\
                \texttt{clock\_speed}&     40.0\\
                \texttt{energy\_gap\_time}&     0.3\\
                \texttt{trigger\_peak\_time}&     0.05\\
                \texttt{trigger\_gap\_time}&     0.0\\
                \texttt{baseline\_length}&     512.0\\
                \texttt{trigger\_threshold}&     150.0\\
                \texttt{baseline\_threshold}&     120.0\\
                \texttt{energy\_threshold}&     0.0\\
                \texttt{peak\_interval\_offset}&     0.5\\
                \texttt{peak\_sample\_offset}&     0.0\\
                \texttt{max\_width}&     0.4\\
                \texttt{peak\_mode}&     0.0\\
                \texttt{peak\_interval}&     0.5\\
                \texttt{peak\_sample}&     0.0\\
                \texttt{polarity}&     1.0\\
                \texttt{preamp\_value}&     1.0\\
                \texttt{gain}&     4.484848\\
                \texttt{gain\_trim}&     1.05\\
                \texttt{number\_mca\_channels}&     8192.0\\
                \texttt{mca\_bin\_width}&     1.0\\
                \texttt{bytes\_per\_bin}&     3.0\\
                \texttt{adc\_trace\_wait}&     0.025\\
                \texttt{auto\_adjust\_offset}&     1.0\\
                \texttt{number\_of\_scas}&     0.0\\
                \bottomrule
            \end{tabular}
        \end{table}

        The parameter \texttt{gain} is set according to \cref{eq:gain calc}\cite{Manual.HandelAPIManual.Xiang}

        \begin{align}
            gain = \frac{1184}{DynRange \cdot PreampValue}
            \label{eq:gain calc}
        \end{align}

        with \(DynRange = \qty{40}{\kV}\) and \(PreampValue = \qty{6.6}{\milli\volt\per\kV}\).
        The latter is taken from the technical specification report where it is referred to as ``analog signal - gain''.
        \texttt{gain\_trim} was adjusted until the K-lines from the brass sample (copper and zinc) aligned with their respective peaks.
        Hereafter, the calibration was checked against a steel sample which is expected to expose itself as two peaks at irons \(K_{\alpha}\) and \(K_{\beta}\) lines and a lead sample.
        Within the energy range, lead should exhibit peaks at its \(L_{\alpha}\) and \(L_{\beta}\) lines.

        \begin{figure}[h]
            \centering
            \includesvg[width=\textwidth]{spectra/spectra_brass_steel_lead}
            \caption[Spectra taken from brass, lead and steel]{Spectra taken from brass, lead and steel at \qty{40}{\kV} using the acquisition values specified in \cref{tab:default detection parameters}.}%
            \label{fig:spectra 40kV}
        \end{figure}

        \Cref{fig:spectra 40kV} shows a combined plot of the spectra taken from brass, steel and lead at an acceleration voltage of \qty{40}{\kV} and a relatively low filament current of \qty{10}{\uA}.
        The full spectrum as read from the MCA is shown at the top and a detailed view of the highlighted region from \qtyrange{5}{20}{\kV} is shown below.
        With a rough calibration made, almost all peaks align good with the emission lines expected from the sample materials.
        % The \textit{microDXP} allows hardware based calibration and storage of value sets in its non-volatile memory as \texttt{parset}s and \texttt{globset}s.
        The spectra being in good agreement with the literature, the setup around the detector is considered functional.\par\medskip
        
        All characteristic energies used are taken from the online dataset provided by the \textsc{National Institute of Standards and Technology}\cite{Dataset.XRayTransitionEnergies.2005}.

    \section{Angular Resolution}\label{sec:angular resolution}
        The theoretical angular resolution for each individual stage is expressed in units of \unit{\degree\per\step} due to the incremental nature of the motion system.
        In general, it can be written as \cref{eq:angular resolution}:

        \begin{align}
            res_{\angle} &= \frac{360}{microsteps \cdot fullStepsPerRevolution} \cdot gearRatio
            \label{eq:angular resolution}
        \end{align}

        For the detector stage this 

        \begin{align}
            res_{\varphi} = \frac{\qty{360}{\degree}}{32 \cdot \qty{200}{\step}} \cdot \frac{80}{250} = \qty{0.018}{\degree\per\step}
        \end{align}

        for the detector stage and

        \begin{align}
            res_{\theta} = \frac{\qty{360}{\degree}}{32 \cdot \cdot \qty{200}{\step}} \cdot \frac{20}{80} \approx \qty{0.0141}{\degree\per\step}
        \end{align}